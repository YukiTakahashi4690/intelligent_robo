%!TEX root = ../thesis.tex

\section{はじめに}
軌道計画とは,制御対象となるシステムやロボットの動きを指定した軌道(経路)に沿って制御するための計画を立てるプロセスである.特に,バンバン制御ではオンオフの制御信号を使用して制御を行う.バンバン制御における軌道計画では,目標軌道に沿ってシステムが移動するように制御信号を生成します.制御対象が目標軌道よりも高い位置にある場合はオフ状態(制御信号がゼロ)であり,制御対象が目標軌道よりも低い位置にある場合はオン状態(最大の制御信号)となる.つまり,バンバン制御はシステムの状態が目標値を超えたり下回ったりするたびに,制御入力が急激に変化することになる. \\

次に多次元空間でのマニピュレータの軌道計画について考える.ここでの軌道とは,各自由度に対して位置,速度,及び加速度の時間履歴を指している. 
