%!TEX root = ../thesis.tex

本研究では, 経路追従行動をカメラ画像を入力としたend-to-end学習で模倣する従来手法を基に, 目標経路上及びその周辺でデータを収集してオフラインで訓練する手法を提案した. 実験では, 経路周辺のデータを多く収集し, バッチ学習を用いて訓練を行った. これにより, テストフェーズでは成功率が100\%となり, 手法の有効性を示すことができた. また, 従来手法では訓練時間が最低でも40分必要であったのに対して, 提案手法を用いることで6分程度で訓練を終了することができた. 結果として, 訓練時間を85\%削減できることを確認した.  